\section{Diskussion}
\label{sec:Diskussion}
Die Theoriewerte für $c_\text{k}$ befinden sich in Tab. \ref{tab:vergleich}. \cite{konstanten} \\ 
Die Molwärme beträgt nach der Theorie $3R \approx \SI{24.94}{\frac{\joule}{\mol\kelvin}}$.
Zudem sind in Tab. \ref{tab:vergleich} die experimentell gemessenenen Werte und deren Abweichungen zu finden.

\begin{table}
    \centering
    \begin{tabular}{c|S[table-format=1.4+-1.3] S[table-format=1.3] S[table-format=2.1]}
        \toprule
        Material & $c_\text{k,exp} / \si{\frac{\joule}{\gram\kelvin}}$ & $c_\text{k,th} / \si{\frac{\joule}{\gram\kelvin}}$ & \text{Abw.} $c_\text{k} / \% $ \\
        \midrule
        Graphit & $\SI{0.80(6)}{}$ & 0.710   & 12.7  \\
        Blei & $\SI{0.1233(31)}{}$ & 0.131  & 5.9 \\
        Aluminium & $\SI{0.840}{}$ & 0.896 & 6.3 \\
        \bottomrule
    \end{tabular}
    \caption{Theorie- und Praxiswerte der Wärmekapazität $c_\text{k}$ und Molwärme $C_\text{V}$ im Vergleich.}
    \label{tab:vergleich}
\end{table}

\begin{table}
    \centering
    \begin{tabular}{c | S[table-format=1.4+-1.3] S[table-format=2.1]}
        \toprule
        Material & $C_\text{V,exp} / \si{\frac{\joule}{\mol\kelvin}}$ & \text{Abw.} $C_\text{V,exp} / \si{\frac{\joule}{\mol\kelvin}}$ \\
        \midrule
        Graphit & $\SI{9.6(7)}{}$ & 61.6 \\
        Blei & $\SI{23.8(7)}{}$ & 4.5 \\
        Aluminium & $\SI{10.04}{}$ & 59.7 \\
        \bottomrule
    \end{tabular}
\caption{Theorie- und Praxiswerte der Molwärme $C_\text{V}$ im Vergleich.}
\label{tab:vergleich_cv}
\end{table}
    