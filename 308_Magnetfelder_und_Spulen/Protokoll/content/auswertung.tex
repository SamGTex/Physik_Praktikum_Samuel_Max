\section{Auswertung}
\label{sec:Auswertung}

\subsection{Lange Spule}

\FloatBarrier

\begin{table}
\centering
\caption{Messwerte der langen Spule.}
\begin{tabular}{cc|cc}
  \toprule
  $x \:/\: \si{\centi\meter}$ & $B \:/\: \si{\milli\tesla}$ & $x \:/\: \si{\centi\meter}$ & $B \:/\: \si{\milli\tesla}$ \\
  \midrule
  -2.5 & 0.134 &  6 & 2.148  \\
  -2 & 0.162 & 12.5 & 2.203 \\
  -1.5 & 0.2 &  13 & 2.182 \\
  -1 & 0.254 & 13.5 & 2.160 \\
  -0.5 & 0.334 &  14 & 2.129 \\ 
  0 & 0.451  & 14.5 & 2.094 \\
  0.5 & 0.601 & 15 & 2.026 \\
  1 & 0.821  & 15.5 & 1.937 \\
  1.5 & 1.085  & 16 & 1.823 \\
  2 & 1.370  & 16.5 & 1.667 \\
  2.5 & 1.610  &  17 & 1.454 \\
  3 & 1.779  & 17.5 & 1.174 \\
  3.5 & 1.904  & 18 & 0.896 \\
  4 & 1.988 & 19 & 0.455 \\
  4.5 & 2.048  &  19.5 & 0.331 \\
  5 & 2.092  & 20 & 0.106 \\
  5.5 & 2.120 & & \\
  \bottomrule
\end{tabular}
\label{tab:long}
\end{table}

\begin{figure}
  \centering
  \caption{Das Magnetfeld der langen Spuele in Abhängikeit vom Ort in der Spule.}
  \includegraphics[width=\textwidth]{content/data/plot_long.pdf}
  \label{fig:long}
\end{figure}

In der Tabelle \ref{tab:long} sind die gemessen Werte aufgefasst.
Dabei steht $x=0$ für den Anfang der Spule.
In der Grafik \ref{fig:long} sind die Werte die in der Tabelle \ref{tab:long} aufgefasst sind, veranschaulicht dargestellt.
Da aus technsichen Gründen das Magnetfeld in der Mitte der Spule nicht gemessen werden konnte, wird Näherungsweise der Messerte bei $6 \si{\centi\meter}$ genommen.
\begin{equation*}
  B(6\si{\centi\meter}) = 2.148 \si{\milli\tesla} 
\end{equation*}

Der nach \eqref{eqn:spule} berechnete Werte liegt bei:
\begin{equation}
  B = 1.9936 \si{\milli\tesla} 
\end{equation}

\FloatBarrier
\subsection{kurze Spule}

\begin{table}
\centering
\caption{Messwerte der kurzen Spule.}
\begin{tabular}{cc|cc}
\toprule
$x \:/\: \si{\centi\meter}$ & $B \:/\: \si{\milli\tesla}$ &$x \:/\: \si{\centi\meter}$ & $B \:/\: \si{\milli\tesla}$ \\
\midrule
-2.5 & 0.079 & 4.5 & 1.817  \\
-2 & 0.104 & 5 & 1.751  \\
-1.5 & 0.177 & 5.5 & 1.634  \\
-1 & 0.239 & 6 & 1.451  \\
-0.5 & 0.326 & 6.5 & 1.196  \\
0 & 0.465  & 7 & 0.937  \\
0.5 & 0.650  & 7.5 & 0.678  \\
1 & 0.905  & 8 & 0.488  \\
1.5 & 1.194  &  8.5 & 0.343  \\
2 & 1.430  & 9 & 0.237  \\
2.5 & 1.618  & 9.5 & 0.181  \\
3 & 1.750  & 10 & 0.137 \\
3.5 & 1.818  & 10.5 & 0.110 \\
4 & 1.837  & 11 & 0.087 \\
\bottomrule
\label{tab:short}
\end{tabular}
\end{table}

\begin{figure}
  \centering
  \caption{Die Messwerte aus der Tabelle \ref{tab:short} grafisch aufgefasst.}
  \includegraphics[width=\textwidth]{content/data/plot_short.pdf}
  \label{fig:short}
\end{figure}

Die Messwerte des Magnetfeldes der kurzen Spule sind in der Tabelle \ref{tab:short} zu sehen.
Die Werte sind anschließend in der Grafik \ref{fig:short} aufgetragen worden.
Das Magnetfeld in der Mitte der Spule hat nach Messung folgenden Wert
\begin{equation*}
  B(4\si{\centi\meter})=1.837\si{\milli\tesla} .
\end{equation*}
Der theoretisch berechnete Werte liegt wiederum bei
\begin{equation*}
  B = 1.477\si{\milli\tesla}.
\end{equation*}

\FloatBarrier

\subsection{Helmholtzspulen}

\begin{table}
\centering
\caption{Messwerte der Helmholtzspulen mit Abstand $d_1=11\si{\centi\meter}$.}
\begin{tabular}{cc}
\toprule
$x \:/\: \si{\centi\meter}$ & $B \:/\: \si{\milli\tesla}$ \\
\midrule
1 & 2.847 \\
1.5& 2.767\\
2& 2.701\\
2.5& 2.646\\
3& 2.610\\
3.5& 2.594\\
4& 2.601\\
4.5& 2.638\\
5& 2.690\\
5.5& 2.755\\
6& 2.836\\
12& 2.356\\
13& 2.002\\
14& 1.640\\
15& 1.336\\
16& 1.077\\
\bottomrule
\label{tab:helm1}
\end{tabular}
\end{table}

\begin{table}
\centering
\caption{Messwerte der Helmholtzspulen mit Abstand $d_2=7\si{\centi\metre}$.}
\begin{tabular}{cc}
\toprule
$x \:/\: \si{\centi\metre}$ & $B \:/\: \si{\milli\tesla}$ \\
\midrule
1& 3.836  \\
1.1& 3.838\\
1.2& 3.837\\
1.3& 3.838\\
1.4& 3.837\\
1.5& 3.838\\
1.6& 3.839\\
1.7& 3.839\\
1.8& 3.840\\
1.9& 3.839\\
2& 3.841\\
2.1& 3.841\\
2.2& 3.842\\
2.3& 3.846\\
8& 2.534\\
9& 2.194\\
10& 1.781\\
11& 1.463\\
12& 1.188\\
\bottomrule
\label{tab:helm2}
\end{tabular}
\end{table}


\begin{table}
\centering
\caption{Messwerte der Helmholtzspulen mit Abstand $d_3=15\si{\centi\metre}$.}
\begin{tabular}{cc}
\toprule
$x \:/\: \si{\centi\metre}$ & $B \:/\: \si{\milli\tesla}$ \\
\midrule
1& 2.423  \\
2& 2.155\\
3& 1.927\\
4& 1.753\\
5& 1.668\\
6& 1.665\\
7& 1.753\\
8& 1.919\\
9& 2.149\\
10& 2.418\\
16& 2.229\\
17& 1.899\\
18& 1.561\\
19& 1.262\\
20& 1.015\\
\bottomrule
\label{tab:helm3}
\end{tabular}
\end{table}

\begin{figure}
\centering
\includegraphics{content/data/plot_helmholtz1.pdf}
\caption{Graphik zu den Messwerten des Abstands $d_1=11\si{\centi\metre}$. }
\label{fig:helm1}
\end{figure}

\begin{figure}
\centering
\includegraphics{content/data/plot_helmholtz2.pdf}
\caption{Graphik zu den Messwerten des Abstands  $d_2=7\si{\centi\metre}$.}
\label{fig:helm2}
\end{figure}

\begin{figure}
\centering
\includegraphics{content/data/plot_helmholtz3.pdf}
\caption{Graphik zu den Messwerten des Abstands $d_3=15\si{\centi\metre}$.}
\label{fig:helm3}
\end{figure}


Das Magnetfeld der Helmholtzspulen wurde üfr drei verschiedene Abstände gemessen.
Die Abstände wurden als $d_1=11\si{\centi\metre}, d_2=7\si{\centi\metre}$ und $d_3= 15 \si{\centi\metre}$ gewählt.
Dabei ergaben die Messung für den Abstand $d_1$ die Werte in Tabelle \ref{tab:helm1}.
Die Werte in Tabelle \ref{teb:helm2} sind bei den Messung für Abstand $d_2=7\si{\centi\metre}$ aufgenommen worden.
Die Messung bei Abstand $d_3=15\si{\centi\metre}$ ergaben die Werte in Tabelle \ref{tab:helm3}.
Die Werte in den Tabellen wurden in den Graphiken \ref{fig:helm1} für $d_1$, \ref{fig:helm2} für $d_2$ und \ref{fig:helm3} für $d_3$ dargstellt.

\subsection{Hysteresekurve}
