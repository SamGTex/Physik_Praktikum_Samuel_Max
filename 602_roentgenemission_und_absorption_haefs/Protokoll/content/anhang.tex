\section{Anhang}
%bragg
\begin{table}
    \centering
    \csvreader[tabular=cc,
    head=false,
    table head= $\theta / \si{\degree}$ & $N_\text{Al} / \si{\frac{Imp}{\second}}$ \\
    \midrule,
    late after line= \\]
    {content/data/Bragg.csv}{1=\eins, 2=\zwei}{$\num{\eins}$ & $\num{\zwei}$}
    \caption{Messdaten zur Überprüfung der Bragg-Bedingung.}
    \label{tab:bragg}  
\end{table}

%Emissionsspektrum
\begin{table}
\centering
    \begin{tabular}{S S}
    \toprule
    $\theta \, / \, \si{\degree}$ & $N \, / \, \si{\frac{Imp}{\second}}$ \\
    \midrule
    8.0 & 323.0\\
    8.1 & 316.0\\
    8.2 & 326.0\\
    8.3 & 340.0\\
    8.4 & 335.0\\
    8.5 & 343.0\\
    8.6 & 350.0\\
    8.7 & 350.0\\
    8.8 & 366.0\\
    8.9 & 357.0\\
    9.0 & 371.0\\
    9.1 & 371.0\\
    9.2 & 372.0\\
    9.3 & 364.0\\
    9.4 & 381.0\\
    9.5 & 379.0\\
    9.6 & 393.0\\
    9.7 & 375.0\\
    9.8 & 391.0\\
    9.9 & 395.0\\
    10.0 & 402.0 \\
    10.1 & 405.0 \\
    10.2 & 390.0 \\
    10.3 & 398.0 \\
    10.4 & 400.0 \\
    10.5 & 418.0 \\
    10.6 & 401.0 \\
    10.7 & 410.0 \\
    10.8 & 408.0 \\
    10.9 & 409.0 \\
    \bottomrule
    \end{tabular}
    \begin{tabular}{S S}
    \toprule
    $\theta \, / \, \si{\degree}$ & $N \, / \, \si{\frac{Imp}{\second}}$ \\
    \midrule
    11.0 & 414.0 \\
    11.1 & 420.0 \\
    11.2 & 417.0 \\
    11.3 & 417.0 \\
    11.4 & 409.0 \\
    11.5 & 406.0 \\
    11.6 & 404.0 \\
    11.7 & 405.0 \\
    11.8 & 400.0 \\
    11.9 & 383.0 \\
    12.0 & 389.0 \\
    12.1 & 382.0 \\
    12.2 & 372.0 \\
    12.3 & 376.0 \\
    12.4 & 385.0 \\
    12.5 & 384.0 \\
    12.6 & 382.0 \\
    12.7 & 373.0 \\
    12.8 & 376.0 \\
    12.9 & 373.0 \\
    13.0 & 375.0 \\
    13.1 & 366.0 \\
    13.2 & 354.0 \\
    13.3 & 341.0 \\
    13.4 & 326.0 \\
    13.5 & 318.0 \\
    13.6 & 305.0 \\
    13.7 & 296.0 \\
    13.8 & 286.0 \\
    13.9 & 285.0 \\
    \bottomrule
    \end{tabular}
    \begin{tabular}{S S}
    \toprule
    $\theta \, / \, \si{\degree}$ & $N \, / \, \si{\frac{Imp}{\second}}$ \\
    \midrule
    14.0 & 274.0 \\
    14.1 & 264.0 \\
    14.2 & 266.0 \\
    14.3 & 270.0 \\
    14.4 & 255.0 \\
    14.5 & 255.0 \\
    14.6 & 260.0 \\
    14.7 & 251.0 \\
    14.8 & 250.0 \\
    14.9 & 248.0 \\
    15.0 & 253.0 \\
    15.1 & 257.0 \\
    15.2 & 248.0 \\
    15.3 & 242.0 \\
    15.4 & 249.0 \\
    15.5 & 246.0 \\
    15.6 & 252.0 \\
    15.7 & 236.0 \\
    15.8 & 234.0 \\
    15.9 & 231.0 \\
    16.0 & 215.0 \\
    16.1 & 217.0 \\
    16.2 & 227.0 \\
    16.3 & 214.0 \\
    16.4 & 217.0 \\
    16.5 & 210.0 \\
         &       \\
         &       \\
         &       \\
         &       \\
    \bottomrule 
    \end{tabular}
\caption{Das Cu-Emissionsspektrum bei einer Beschleunigungsspannung von $U=\SI{35}{\kilo\volt}$ und $I=\SI{1}{\milli\ampere}$.}
\label{tab:spektrum1}
\end{table}

%Zink, Gallium
\begin{table}
\centering
    \begin{tabular}{S S}
    \toprule
    $\theta \, / \, \si{\degree}$ & $N_\text{Zn} \, / \, \si{\frac{Imp}{\second}}$ \\
    \midrule
    18.0 & 58.0 \\
    18.1 & 54.0 \\
    18.2 & 55.0 \\
    18.3 & 54.0 \\
    18.4 & 54.0 \\
    18.5 & 55.0 \\
    18.6 & 65.0 \\
    18.7 & 84.0 \\
    18.8 & 91.0 \\
    18.9 & 100.0 \\
    19.0 & 102.0 \\
    19.1 & 100.0 \\
    19.2 & 98.0 \\
    19.3 & 100.0 \\
    19.4 & 95.0 \\
    19.5 & 98.0 \\
         &      \\
         &      \\
         &      \\
         &      \\
         &      \\
    \bottomrule
    \end{tabular}
    \begin{tabular}{S S}
    \toprule
    $\theta \, / \, \si{\degree}$ & $N_\text{Ga} \, / \, \si{\frac{Imp}{\second}}$ \\
    \midrule
    17.0 & 66.0 \\
    17.1 & 66.0 \\
    17.2 & 78.0 \\
    17.3 & 88.0 \\
    17.4 & 102.0 \\
    17.5 & 116.0 \\
    17.6 & 121.0 \\
    17.7 & 121.0 \\
    17.8 & 122.0 \\
    17.9 & 122.0 \\
    18.0 & 119.0 \\
    18.1 & 114.0 \\
    18.2 & 110.0 \\
    18.3 & 108.0 \\
    18.4 & 104.0 \\
    18.5 & 110.0 \\
    18.6 & 110.0 \\
    18.7 & 109.0 \\
    18.8 & 99.0 \\
    18.9 & 100.0 \\
    19.0 & 98.0 \\
    \bottomrule
    \end{tabular}
    \caption{Absorptionsspektrum von Zink (Zn) und Gallium (Ga).}
    \label{tab:znba}
\end{table}

%Brom, Rubidium
\begin{table}
\centering
    \begin{tabular}{S S}
    \toprule
    $\theta \, / \, \si{\degree}$ & $N_\text{Br} \, / \, \si{\frac{Imp}{\second}}$ \\
    \midrule
    12.8 & 10.0 \\
    12.9 & 12.0 \\
    13.0 & 9.0 \\
    13.1 & 13.0 \\
    13.2 & 18.0 \\
    13.3 & 21.0 \\
    13.4 & 25.0 \\
    13.5 & 27.0 \\
    13.6 & 27.0 \\
    13.7 & 22.0 \\
    13.8 & 25.0 \\
    13.9 & 21.0 \\
    14.0 & 23.0 \\
    14.1 & 20.0 \\
    14.2 & 21.0 \\
    14.3 & 19.0 \\
    \bottomrule
    \end{tabular}
    \begin{tabular}{S S}
    \toprule
    $\theta \, / \, \si{\degree}$ & $N_\text{Rb} \, / \, \si{\frac{Imp}{\second}}$ \\
    \midrule
    11.2 & 11.0 \\
    11.3 & 10.0 \\
    11.4 & 10.0 \\
    11.5 & 12.0 \\
    11.6 & 17.0 \\
    11.7 & 32.0 \\
    11.8 & 39.0 \\
    11.9 & 47.0 \\
    12.0 & 57.0 \\
    12.1 & 64.0 \\
    12.2 & 61.0 \\
    12.3 & 57.0 \\
    12.4 & 54.0 \\
    12.5 & 54.0 \\
         &      \\
         &      \\
    \bottomrule
    \end{tabular}
    \caption{Absorptionsspektrum von Brom (Br) und Rubidium (Rb).}
    \label{tab:brrb}
\end{table}

%Strontium, Zirkonium
\begin{table}
\centering
    \begin{tabular}{S S}
    \toprule
    $\theta \, / \, \si{\degree}$ & $N_\text{Sr} \, / \, \si{\frac{Imp}{\second}}$ \\
    \midrule
    10.5 & 43.0 \\
    10.6 & 41.0 \\
    10.7 & 40.0 \\
    10.8 & 44.0 \\
    10.9 & 50.0 \\
    11.0 & 89.0 \\
    11.1 & 120.0 \\
    11.2 & 152.0 \\
    11.3 & 181.0 \\
    11.4 & 193.0 \\
    11.5 & 181.0 \\
    11.6 & 196.0 \\
    11.7 & 181.0 \\
    11.8 & 173.0 \\
    11.9 & 166.0 \\
    12.0 & 159.0 \\
    \bottomrule
    \end{tabular}
    \begin{tabular}{S S}
    \toprule
    $\theta \, / \, \si{\degree}$ & $N_\text{Zr} \, / \, \si{\frac{Imp}{\second}}$ \\
    \midrule
    9.5 & 112.0 \\
    9.6 & 120.0 \\
    9.7 & 126.0 \\
    9.8 & 147.0 \\
    9.9 & 180.0 \\
    10.1 & 266.0 \\
    10.2 & 282.0 \\
    10.3 & 290.0 \\
    10.4 & 301.0 \\
    10.5 & 295.0 \\
    10.6 & 283.0 \\
    10.7 & 296.0 \\
    10.8 & 283.0 \\
    10.9 & 286.0 \\
    11.0 & 286.0 \\
         &       \\
    \bottomrule
    \end{tabular}
    \caption{Absorptionsspektrum von Strontium (Sr) und Zirkonium (Zr).}
    \label{tab:srzr}
\end{table}