\input{header.tex}

\subject{802}
\title{Fourier Synthese}
\date{%
  Durchführung: 22.10.2019
  \hspace{3em}
  Abgabe: 29.10.2019
}

\begin{document}

\maketitle
\thispagestyle{empty}
\tableofcontents
\newpage
%\printbibliography{}
\newpage
\section{Theorie}
Jede periodische Funktion läßt sich in eine Reihe aus sin- und cos-Termen entwickeln (Fourierreihe)
\begin{equation}
  f(t)=\sum_{i=0}^\infty (A_k\cdot cos(\omega_k t) + B_k\cdot sin(\omega_k t))
\end{equation}
mit
\begin{equation}
\omega_k=\frac{2\pi k}{T}
\label{eqn:omega}
\end{equation}
\section{Fourier-Zerlegung der Funktion |sin(x)|}
\subsection{Berechnung der Integrale}
Im folgenden soll die Funktion $f(x)=|sin(x)|$ mit einer Fourierreihe angenähert werden.
Die Koeffizienten sind definiert als
\begin{equation}
A_k=\frac{2}{T} \int_{-T/2}^{T/2} f(t) \cdot cos(\omega_k \cdot t)  \, \symup{d}t
\end{equation}
\begin{equation}
B_k=\frac{2}{T} \int_{-T/2}^{T/2} f(t) \cdot sin(\omega_k \cdot t)  \, \symup{d}t
\end{equation}
Die Funktion $|sin(x)|$ erfüllt die Eigenschaft $f(-x) = f(x)$ und ist somit eine gerade Funktion.
Somit ist unser $B_k=0$. 
Zunächst wählen wir $T=2\pi$. Somit ist $\omega_k =k$.
Daraus folgt für $A_k$
\begin{equation}
A_k=\frac{1}{\pi} \int_{-\pi}^{\pi} |sin(t)| \cdot cos(\omega_k \cdot t) \, \symup{d}t
\end{equation}
Die Funktion $|sin(x)|$ ist $\pi$-periodisch und hat bei $x = 0$ eine Nullstelle.
Wir integrieren nun über eine, statt wie zuvor über zwei Perioden. Es gilt 
$\int_{-\pi}^{0}f(x) \, \symup{d}x = \int_0^{\pi} f(x) \, \symup{d}x$
erhalten wir zusätlich einen Faktor $2$.
\begin{equation}
A_k=\frac{2}{\pi} \int_0^{\pi} |sin(t)| \cdot cos(\omega_k \cdot t) \, \symup{d}t
\end{equation}
Es gilt $|sin(x)| = sin(x)$ im Intervall $I=[0,\pi]$, so können wir
unser Integral vereinfachen
\begin{equation}
A_k=\frac{2}{\pi} \int_0^{\pi} sin(t) \cdot cos(\omega_k \cdot t)  \, \symup{d}t
\end{equation}
\begin{equation}
\Rightarrow A_k= \frac{2}{\pi} \left [ \frac{cos((\omega_k-1)t)}{2(w-1)} - \frac{cos((\omega_k+1)t)}{2(\omega_k^2+1)} \right ]_0^{\pi}
\end{equation}
mit $\omega_k$ und Grenzen eingesetzt erhalten wir
\eqref{eqn:omega}
\begin{equation}
A_k=\frac{2}{\pi}\left ( \frac{cos(2k\pi-\pi)}{4k-2} - \frac{cos(2k\pi+\pi)}{4k+2} - \frac{cos(0)}{4k-2} + \frac{cos(0)}{4k+2} \right )
\end{equation}
\begin{equation}
\Leftrightarrow A_k=\frac{2}{\pi}\left ( \frac{-1}{4k-2} + \frac{1}{4k+2} - \frac{1}{4k-2} + \frac{1}{4k+2} \right )
\end{equation}
\begin{equation}
\Leftrightarrow A_k= \frac{2}{\pi} \left ( \frac{-8}{(4k-2)(4k+2)} \right )
\end{equation}
\begin{equation}
\Leftrightarrow A_k= \frac{4}{-4k^2 \pi + \pi}
\end{equation}
\subsection{Tabelle}
Für das online-Experiment werden 17 Koeffizienten benötigt um die
Regler einzustellen 
%tabelle
\begin{table}
  \centering
  \label{tab:koeffizienten}
  \begin{tabular}{c c}
    \toprule
    $k$ & $A_k$\\
    \midrule
      0 & 1.27323954e+00\\
      1 & -4.24413182e-01\\ 
      2 & -8.48826363e-02\\ 
      3 & -3.63782727e-02\\
      4 & -2.02101515e-02\\
      5 & -1.28610055e-02\\ 
      6 & -8.90377304e-03\\
      7 & -6.52943356e-03\\
      8 & -4.99309625e-03\\
      9 & -3.94191810e-03\\
      10 & -3.19107655e-03\\
      11 & -2.63610672e-03\\
      12 & -2.21432964e-03\\
      13 & -1.88628081e-03\\
      14 & -1.62610414e-03\\
      15 & -1.41628425e-03\\
      16 & -1.24461344e-03\\
      17 & -1.10237190e-03\\
    \bottomrule
  \end{tabular}
\end{table}
\newpage
\subsection{Plot}
Wir erhalten als Ergebnis auf der
\href{https://www.j-berkemeier.de/Fouriersynthese.html}{Website}
folgende Grafik
\begin{figure}
  \centering
  \includegraphics[height=10cm]{content/screenshot_phase90}
  \caption{Fouriersynthese von $|sin(x)|$}
  \label{fig:sin}
\end{figure}
\end{document}