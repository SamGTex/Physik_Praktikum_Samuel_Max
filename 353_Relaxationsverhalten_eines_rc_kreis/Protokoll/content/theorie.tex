\section{Theorie}
\label{sec:Theorie}


\subsection{Zielsetzung}

\subsection{Relaxion eines RC-Kreises}
Ein RC-Kreis wird durch aufladen oder entladen zur Relaxion gebracht.
Zum Zeitpunkt $t$ ist die Änderungsrate der betrachteten physikalischen Größe $A$ dann

\begin{equation}
    \frac{dA}{dt} = c \left[ A(t) - A(\inf)  \right]  .
    \label{eqn:diffA}
\end{equation}
Aus der Gleichung folgt 

\begin{equation}
    A(t) = A(\inf) + \left[ A(0) - A(\inf)  \right] e^{ct} .
\end{equation}

\subsubsection{Entladevorgang}

Nun wird auf den Entladevorgang eines RC-Kreises eingegangen.
Nach dem ohmschen Gesetzt gilt 

\begin{equation}
    I = \frac{U_\text{C}}{R}, 
    \label{eqn:ohm}
\end{equation}
wobei $U_\text{C}$ die Spannung am Kondensator ist, welche durch
\begin{equation}
    U_\text{C} = \frac{Q}{C}
    \label{eqn:spannungkondi}
\end{equation}
berechnet werden kann.
In einem gewissen Zeitintervall $dt$ fließt durch den Leiter eine Ladungsmenge $Idt$.
Dadurch ändert sich die Ladung der Kondensatorplatten um
\begin{equation}
    dQ = -Idt.
    \label{eqn:dQ}
\end{equation}

Da die Spannung am Kondensator $U_\text{C}$ unbekannt ist wird diese mithilfe von Gleichung \eqref{eqn:ohm}, \eqref{eqn:spannungkondi} und \eqref{eqn:dQ} eliminiert.
Durch umformen ergibt sich schließlich für die zeitliche Änderung der Ladung am Kondensator 
\begin{equation*}
    \frac{dQ}{dt} = - \frac{1}{RC} Q(t).
\end{equation*}
Diese Gleichung hat die Form wie Gleichung \eqref{eqn:diffA}.
Da sich nach unedlich langer Zeit keine Ladung mehr auf dem Kondensator befindet, wenn dieser nicht aufgeladen wird, gilt 

\begin{equation*}
    Q(\inf) = 0.
\end{equation*}
Daraus folgt schließlich die Formel
\begin{equation}
    Q(t) = Q(0) e^{-\frac{t}{RC}}
    \label{eqn:ladungkondi}
\end{equation}
welche die auf dem Kondensator befindliche Ladung zum Zeitpunkt $t$ wiedergibt.

\subsection{Spannungsänderung durch Relaxion}
Im allgemeinen gilt für Wechselspannung 
\begin{equation}
    U(t) = U_0 \cos(\omega t).
\end{equation}
Für die Wechselspannung am Kondensator gilt 
\begin{equation}
    U_\text{C}(t) = A(\omega) \cos(\omega t + \phi(\omega)).
\end{equation}