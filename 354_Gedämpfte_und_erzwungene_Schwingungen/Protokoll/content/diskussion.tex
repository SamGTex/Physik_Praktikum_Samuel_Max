\section{Diskussion}
\label{sec:Diskussion}

Die aufgenommenen Werte führen alle zu guten Diagrammen die den erwarteten Verläufen folgen.
Der effektive Widerstand der experimentell bestimmt wird liegt bei $R_1 = \SI{136.176(403)}{\ohm}$ der Widerstand, welcher verbaut wurde hat aber einen Wert von $R_{1,\text{real}}=\SI{67.2(1)}{\ohm}$.
Die Differenz kann mit dem Innenwiderstand des Generators, von $\SI{50}{\ohm}$ begründet werden.
Die Abweichungen vom gegeben Wert und dem experimentell bestimmten Wert ist auch in der Tabelle \ref{tab:fehler} zu finden.
In der selben Tabelle ist auch die Abweichung vom experimentell bestimmten Widerstand für den aperiodischen Grenzfall und der theoretische Wert für den selben aufgefasst.
Die beiden Werte haben eine Abweichung von 12.59\% voneinander.
In der dritten Zeile ist die Güte $q$ zu finden, dabei hat der experimentelle Wert und theoretische Wert eine recht kleine Abweichung.
In der letzte Zeile ist die Breite der Resonanzkurve zu finden, dabei hat der experimentelle Wert eine Abweichung von 15 \% zum theoretischen Wert.

\begin{table}
\centering
\caption{Die relativen Abweichungen der Messwerte von den gegebene Werten.}
\begin{tabular}{cccc}
\toprule
 & Messwert & gegebener Wert & Abweichung $ \,/\, \%$ \\
\midrule
$R_1$ &  $\SI{136.176(403)}{\ohm}$ & $\SI{67.2(1)}{\ohm}$ & 28.23 \\
$R_ap$ & $\SI{5}{\kilo\ohm}$ & $\SI{5.72(4)}{\kilo\ohm}$ & 12.59 \\
$q$ & 3.657 & $\SI{3.909(029)}{}$ & 6.45 \\
$f_{+}- f_{-}$ & $\SI{8}{\kilo\hertz}$ & $\SI{6947(21)}{\hertz}$ & 15,00 \\
\bottomrule
\end{tabular}
\label{tab:fehler}
\end{table}