\input{header.tex}

\subject{802}
\title{Fourier Synthese}
\date{%
  Durchführung: 22.10.2019
  \hspace{3em}
  Abgabe: 29.10.2019
}

\begin{document}

\maketitle
\thispagestyle{empty}
\tableofcontents
\newpage
%\printbibliography{}
\newpage
\section{Theorie}
Jede periodische Funktion läßt sich in eine Reihe aus sin- und cos-Termen entwickeln (Fourierreihe)
\begin{equation}
  f(t)=\sum_{i=0}^\infty (A_k\cdot cos(\omega_k t) + B_k\cdot sin(\omega_k t))
\end{equation}
mit
\begin{equation}
\omega_k=\frac{2\pi k}{T}
\end{equation}
\section{Fourier-Zerlegung der Funktion |sin(x)|}
Im folgenden soll die Funktion $f(x)=|sin(x)|$ mit einer Fourierreihe angenähert werden.
Die Koeffizienten sind definiert als
\begin{equation}
A_k=\frac{2}{T} \int_{-T/2}^{T/2} f(t) \cdot cos(\omega_k \cdot t)  dt
\end{equation}
\begin{equation}
B_k=\frac{2}{T} \int_{-T/2}^{T/2} f(t) \cdot sin(\omega_k \cdot t)  dt
\end{equation}
Die Funktion $|sin(x)|$ erfüllt die Eigenschaft $f(-x) = f(x)$ und ist somit eine gerade Funktion.
Somit ist unser $B_k=0$. 
Zunächst wählen wir $T=2\pi$. Somit ist $\omega_k =k$.
Daraus folgt für $A_k$
\begin{equation}
A_k=\frac{1}{\pi} \int_{-\pi}^{\pi} |sin(t)| \cdot cos(\omega_k \cdot t) dt
\end{equation}
Die Funktion $|sin(x)|$ ist $\pi$-periodisch und hat bei $x = 0$ eine Nullstelle.
Wir integrieren nun über eine Periode. Damit das Ergebnis weiterhin stimmt,
erhalten wir zusätlich einen Faktor $2$.
\begin{equation}
A_k=\frac{2}{\pi} \int_0^{\pi} |sin(t)| \cdot cos(\omega_k \cdot t)  dt
\end{equation}
Es gilt $|sin(x)| = sin(x)$ im Intervall $I=[0,\pi]$, so können wir
unser Integral vereinfachen
\begin{equation}
A_k=\frac{2}{\pi} \int_0^{\pi} sin(t) \cdot cos(\omega_k \cdot t)  dt
\end{equation}
\begin{equation}
\Rightarrow A_k=\left [ \frac{\omega_k sin(t) sin(\omega_k t) + cos(t) cos(\omega_k t)}{\omega_k^2 - 1} \right ]_0^{\pi}
\end{equation}
\end{document}
