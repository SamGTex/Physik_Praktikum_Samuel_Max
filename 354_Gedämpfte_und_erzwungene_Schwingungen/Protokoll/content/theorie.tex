\section{Theorie}
\label{sec:Theorie}

\subsection{LC-Kreis}
\label{sec:lckreis}
Der LC-Kreis enthält eine Spule mit Induktivität $L$ und einen Kondensator mit Kapazität $C$ (siehe Abb. \ref{fig:lckreis}).
Die Energie pendelt zwischen den beiden Energiespeichern Spule und Kondensator hin und her.
Hier handelt es sich um einen ungedämpften Oszillator, da die Gesamtenergie im System vollständig erhalten bleibt.

\subsection{LRC-Kreis}
In Abbildung \ref{fig:lrckreis} ist der LRC-Kreis abgebildet.
Das System enthält einen Widerstand $R$, Induktivität $L$ und Kapazität $C$.
Die Energie schwingt (wie im LC-Kreis \ref{sec:lckreis}) zwischen Kondensator und Spule hin und her.
Es handelt sich um eine gedämpfte Schwingung, da Energie am Widerstand $R$ in Wärmeenergie umgewandelt wird und die Gesamtenergie im System mit der Zeit abnimmt.

\begin{figure}
    \centering
    \begin{subfigure}{0.48\textwidth}
        \centering
        \includegraphics[height=3cm]{content/data/lckreis.jpg}
        \caption{Ungedämpfter Schwingkreis}
        \label{fig:lckreis}
    \end{subfigure}
    \begin{subfigure}{0.48\textwidth}
        \centering
        \includegraphics[height=3cm]{content/data/lrckreis.jpg}
        \caption{Gedämpfter Schwingkreis} 
        \label{fig:lrckreis}
    \end{subfigure}
    \caption{Schwingkreise mit den Bauelelementen: Kondensator mit Kapazität $C$, Spule mit Induktivität $L$, ohmscher Widerstand $R$.}
\end{figure}