\section{Theorie}
\label{sec:Theorie}

\subsection{Relaxion eines RC-Kreises}
Ein RC-Kreis wird zur Relaxion gebracht.
Zum Zeitpunkt $t$ ist die Änderungsrate der betrachteten physikalischen Größe $A$ dann

\begin{equation}
    \frac{dA}{dt} = c \left[ A(t) - A(\infty)  \right]  .
    \label{eqn:diffA}
\end{equation}
Aus der Gleichung folgt 

\begin{equation}
    A(t) = A(\infty) + \left[ A(0) - A(\infty)  \right] e^{ct} .
\end{equation}

\subsubsection{Entladevorgang}

Nun wird auf den Entladevorgang eines RC-Kreises eingegangen.
Nach dem ohmschen Gesetzt gilt 

\begin{equation}
    I = \frac{U_\text{C}}{R}, 
    \label{eqn:ohm}
\end{equation}
wobei $U_\text{C}$ die Spannung am Kondensator ist, welche durch
\begin{equation}
    U_\text{C} = \frac{Q}{C}
    \label{eqn:spannungkondi}
\end{equation}
berechnet werden kann.
In einem gewissen Zeitintervall $dt$ fließt durch den Leiter eine Ladungsmenge $Idt$.
Dadurch ändert sich die Ladung der Kondensatorplatten um
\begin{equation}
    dQ = -Idt.
    \label{eqn:dQ}
\end{equation}

Durch umformen mit den Gleichungen \eqref{eqn:ohm}, \eqref{eqn:spannungkondi} und \eqref{eqn:dQ} wird nun eine Gleichung für die zeitliche Änderung der Ladung aufgstellt.
\begin{equation*}
    \frac{dQ}{dt} = - \frac{1}{RC} Q(t).
\end{equation*}
Diese Gleichung hat die Form wie Gleichung \eqref{eqn:diffA}.
Da sich nach unedlich langer Zeit keine Ladung mehr auf dem Kondensator befindet, wenn dieser nicht aufgeladen wird, gilt 

\begin{equation*}
    Q(\infty) = 0.
\end{equation*}
Daraus folgt schließlich die Formel
\begin{equation}
    Q(t) = Q(0) e^{-\frac{t}{RC}}
    \label{eqn:ladungkondi}
\end{equation}
welche die auf dem Kondensator befindliche Ladung zum Zeitpunkt $t$ wiedergibt.

\subsection{Spannungsänderung durch Relaxion}
Für Wechselspannung gilt 
\begin{equation}
    U(t) = U_0 \cos(\omega t).
\end{equation}
Für die Wechselspannung am Kondensator gilt 
\begin{equation}
    U_\text{C}(t) = A(\omega) \cos(\omega t + \varphi(\omega)).
\end{equation}
Nach den Kirchoffschen Gesetzten gilt zusätzlich
\begin{equation}
    U(t) = U_\text{R}(t) + U_\text{C}(t).
\end{equation}
Nach Ersetzung der Spannungen folgt schließlich für den Freuquenz abhängigen Phasenunterschied
\begin{equation}
    \varphi(\omega) = \arctan(-\omega RC).
    \label{eqn:phase}
\end{equation}
Mit dem nun bekannten Phasenunterschied zwischen Spannung am Kondensator und der Spannungsquelle lässt sich nun noch die frequenzabhängige Amplitude der Spannung berechnen.
Die Spannungsamplitude $A$ lässt sich durch 
\begin{equation}
    A(\omega) = \frac{U_0}{\sqrt{1+\omega^2 R^2 C^2}}
    \label{eqn:ampli}
\end{equation}
berechnen.
Sie ist, wie in der Formel \ref{eqn:ampli} zu erkennen, frequenzabhängig.
Für eine Frequenz von $\omega = 0$ ist sie gleich der Spannung der Spannungsquelle $U_0$.
Für eine Frequenz von $\omega = \infty$ geht die Amplitude gegen 0.

\subsection{RC-Kreis als Integrator}
Unter bestimmten Vorraussetzung kann ein RC-Kreis eine zeitlich veränderliche Spannung integrieren.
Wenn dies der Fall ist gilt der RC-Kreis als Integrator.
Damit dies der Fall ist muss für die Frequenz $\omega$ gelten
\begin{equation*}
    \omega >> \frac{1}{RC}.
\end{equation*} 
Die Spannung im RC Kreis wird beschrieben durch
\begin{equation}
    U(t) = U_\text{R}+U_\text{C}(t) = R I(t) + U_\text{C}(t).
    \label{eqn:integtator_0}
\end{equation}
Die Gleichung \eqref{eqn:integtator_0} führt zu
\begin{equation*}
    U(t) = RC \frac{dU_\text{C}}{dt}+U_\text{C}
\end{equation*}
oder in integraler Form 
\begin{equation}
    U_\text{C}(t) = \frac{1}{RC} \int_0^t  U(t') \symup{d}t'.
    \label{eqn:integator}
\end{equation}