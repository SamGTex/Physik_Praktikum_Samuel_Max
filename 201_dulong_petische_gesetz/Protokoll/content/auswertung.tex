\section{Auswertung}
\label{sec:Auswertung}
\subsection{Wärmekapazität des Kalorimeters}
Das Wasser mit Masse $m_\text{x}$ und Temperatur $T_\text{x}$ befindet sich im Kalorimetergefäß.
Ein weiteres Gefäß mit Wasser mit der Masse $m_\text{y}$ wird auf die Temperatur $T_\text{y}$ erwärmt.
Der Literaturwert für die Wärmekapazität von Wasser beträgt $c_\text{w} = \SI{4.18}{\joule / \gram\kelvin}$ \cite[159]{anleitung}.
Es wurden folgende Werte gemessen:
\begin{equation*}
    m_\text{x} = \SI{263.94}{\gram}
\end{equation*}
\begin{equation*}
    m_\text{y} = \SI{286.95}{\gram}
\end{equation*}
\begin{equation*}
    T_\text{x} = \SI{294.75}{\kelvin}    
\end{equation*}
\begin{equation*}
    T_\text{y} = \SI{353.15}{\kelvin}
\end{equation*}
\begin{equation*}
    T_\text{m} = \SI{323.15}{\kelvin}
\end{equation*}

Nach Gleichung \eqref{eqn:ref} folgt für die Wärmekapazität
\begin{equation}
    c_\text{g}m_\text{g} = \SI{163.76}{\joule/\kelvin} .
\end{equation}

\subsection{Spezifische Wärmekapazität}
\subsubsection{Graphit}
Die 3 Messungen für die Probe Graphit ergeben die Werte in Tab. \ref{tab:graphit}.
\begin{table}
    \centering
    \csvreader[tabular=c|cccccc,
    head=false,
    table head= Messung & $m_\text{k} / \si{\kg}$ & $m_\text{Gefäß} / \si{\kg}$ & $m_\text{Glas} / \si{\kg}$ & $T_\text{k} / \si{\kelvin}$ & $T_\text{w} / \si{\kelvin}$ & $T_\text{m} / \si{\kelvin}$\\
    \midrule,
    late after line= \\]
    {content/data/graphit.csv}{1=\eins, 2=\zwei, 3=\drei, 4=\vier, 5=\fuenf, 6=\sechs, 7=\sieben}{$\num{\eins}$ & $\num{\zwei}$ & $\num{\drei}$ & $\num{\vier}$ & $\num{\fuenf}$ & $\num{\sechs}$ & $\num{\sieben}$}
    \caption{Die gemessenen Daten zur Probe Graphit. }
    \label{tab:graphit}  
\end{table}
Nach der Gleichung \eqref{eqn:ref} für die spezifische Wärmekapazität des Probekörpers, folgt
\begin{equation}
    c_\text{k,1} = \SI{0.678}{\joule/\gram\kelvin}
\end{equation}
\begin{equation}
    c_\text{k,2} = \SI{0.850}{\joule/\gram\kelvin}
\end{equation}
\begin{equation}
    c_\text{k,3} = \SI{0.876}{\joule/\gram\kelvin}
\end{equation}
Um den Mittelwert zu ermitteln wird die Formel
\begin{equation}
    \mu = \frac{1}{n} \sum_{i=1}^n x_i
    \label{eqn:mittel}
\end{equation}
verwendet. Hier wird das Python-Plugin Numpy \cite{numpy} verwendet.
Wobei $x_i$ der $i$-te Wert bei $n$ Werten ist.
Um den Fehler zu berechnen wird
\begin{equation}
    \sigma = \sqrt{\frac{1}{n-1} \sum_{i=1}^n (x_i - \mu)^2}
    \label{eqn:fehler}
\end{equation}
verwendet. Hier wird das Python-Plugin Numpy \cite{numpy} verwendet.
Werden fehlerbehaftete Größen in Formeln verwendet, so wird im Folgenden die Gauß'sche Fehlerfortpflanzung 
\begin{equation}
    \Delta y = \left|\frac{\partial y}{\partial x_1}\right| \Delta x_1 + \left|\frac{\partial y}{\partial x_2}\right| \Delta x_2 + ...
\end{equation}
verwendet. Die $\Delta$-Werte beschreiben die Fehlergrenzen.
Die Fehler werden im Folgenden mithilfe des Python-Plugin uncertainties \cite{uncertainties} berechnet.
\\
Im Mittel \eqref{eqn:mittel} mit dem Fehler \eqref{eqn:fehler} beträgt die spezifische Wärmekapazität für Blei
\begin{equation}
    c_\text{k,Blei} = \SI{9.6(7)}{\joule/\gram\kelvin} .
\end{equation}
Die folgenden Mittelwerte werden nach Gleichung \eqref{eqn:mittel} mit dem dazugehörigen Fehler \eqref{eqn:fehler} berechnet.

\subsubsection{Blei}
\label{sec:Auswertung}
\begin{table}
    \centering
    \csvreader[tabular=c|cccccc,
    head=false,
    table head= Messung & $m_\text{k} / \si{\kg}$ & $m_\text{Gefäß} / \si{\kg}$ & $m_\text{Glas} / \si{\kg}$ & $T_\text{k} / \si{\kelvin}$ & $T_\text{w} / \si{\kelvin}$ & $T_\text{m} / \si{\kelvin}$\\
    \midrule,
    late after line= \\]
    {content/data/blei.csv}{1=\eins, 2=\zwei, 3=\drei, 4=\vier, 5=\fuenf, 6=\sechs, 7=\sieben}{$\num{\eins}$ & $\num{\zwei}$ & $\num{\drei}$ & $\num{\vier}$ & $\num{\fuenf}$ & $\num{\sechs}$ & $\num{\sieben}$}
    \caption{Die gemessenen Daten zur Probe Blei. }
    \label{tab:blei}  
\end{table}