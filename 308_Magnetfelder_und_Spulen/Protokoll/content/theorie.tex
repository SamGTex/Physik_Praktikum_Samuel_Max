\section{Theorie}
\label{sec:Theorie}
\subsection{Grundlagen}
Magnetfelder werden durch die Bewegung elektrischer Ladungen erzeugt.
Die magnetische Feldstärke $\vec{H}$ beschreibt die Richtung und Stärke des Magnetfelds.
Der Verlauf des Magnetfeldes kann durch Feldlinien dargestellt werden.
Diese bilden sogenannte Schleifen und sind immer geschlossen.
Häufig besitzen Atome allein durch ihre Elektronenanordnung und -bewegung ein dauerhaftes magnetisches Moment.
In dem Fall, dass die magnetischen Momente statistisch verteilt sind, kann die magnetische Flussdichte $\vec{B}$ verwendet werden:
\begin{equation}
    \vec{B} = \mue \cdot \vec{H}
\end{equation}
Dabei ist $\mue$ die materialabhängige Permeabilität und $\vec{H}$ das von außen angelegte Magnetfeld.  