\section{Durchführung}
\label{sec:Durchführung}

Zur Messung der Spannung wird ein digitales Oszilloskop benutzt.

\subsection{Wheatstonesche Brücke}

Zunächst wird die Wheatstonesche Brücke wie in Abbildung \ref{fig:wheatstonesche} dargestellt aufgebaut.
Danach wird ein bekannter Widerstand an der stelle der Widerstands $R_2$ eingesetzt.
Nun wird das Widerstandsverhältnis des Potentiometers, also $R_3$ und $R_4$, solange variiert, bis mit dem Oszilloskop zwischen Punkt $A$ und $B$ keine Brückspannung mehr gemessen wird.
Daraufhin werden die Widerstände für $R_3$ und $R_4$ notiert und $R_2$ durch einen neuen bekannten Widerstand ausgetauscht.
Dieses Prozedere wird drei mal durchgeführt.
Danach wird ein neuer unbekannter Widerstand $R_x$ eingebaut und der Versuch wird wie oben beschrieben erneut durchgeführt.

\subsection{Kapazitätsmessbrücke}

Die Kapazitätsmessbrücke wird nach Abbildung \ref{fig:kapaz} aufgebaut.
In unserem Aufbau ist der Widerstand $R_2$ ein fester bekannter Widerstand.
Nun wird wieder das Widerstandsverhältnis zwischen $R_3$ und $R_4$ variiert, bis keine Brückspannung mehr messbar ist.
Daraufhin werden die Widerstandswerte von $R_3$ und $R_4$ notiert und der Kondensator $C_2$ ausgetauscht.
Dieser Prozess wird zwei mal wiederholt.

\subsection{Induktivitätsmessbrücke}