\section{Diskussion}
\label{sec:Diskussion}

Die aufgenommen Werte führen alle zu guten Diagrammen die den Verläufen folgende, die auch erwartet waren.
Allerdings entsprechen die berechneten Werte bei den meisten Rechnungen, nicht den realen Werten.
So weicht der effektive Widerstand der experimentell bei $R_1 = \SI{0.136164(403)}{\ohm}$ liegen müsst, eigentlich bei 
$R_{1,\text{real}}=\SI{67.2(1)}{\ohm}$ , diese Abweichung entspricht $49352.25 \%$, was viel zu hoch ist.
Diese Fehler kann auch nicht nur mit einem Fehler am Gerät oder kleinen Messungenauigkeiten begründet werden.
Die Berechnung des Wertes ist also Fehlerhaft.
Da der Widerstand eine so große Abweichung vom eigentlichen Wert aufweist lässt sich schließen, dass auch der in \ref{sec:gedaempfte} berechnete Wert für $\mu=0.6423$ falsch sein muss.
Allerdings lässt sich mit diesem Wert eine Ausgleichsfunktion aufstellen die sehr gut zu den aufgenommenen Messwerten passt.
Deswegen ist es uneinsichtlich wo genau der Fehler in der Berechnung liegt.
 