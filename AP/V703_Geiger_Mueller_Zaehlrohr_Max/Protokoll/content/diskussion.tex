\section{Diskussion}
\label{sec:Diskussion}
Die Charakteristik zeigt das zuvor erwartete Bild.
Die absolute Steigung des Plateaus der Charakteristik ist 
\begin{align*}
 a =& \SI{1.16(22)}{\frac{\text{Imp}}{\V}}. \\
\end{align*}
Die relative Steigung ist $a=\SI{1.7(5)}{\percent\per100\volt}$.
Damit ist die Steigung gering genug um eine Messung der Intensität zu ermöglichen.
Allerings sind deutliche statistische Unsicherheiten zu erkennen.
So sind manche Messwerte im Berreich niedriger Spannung schon so groß wie die Werte am rechtem Rand des Plateaus.
\\\\
Die Totzeit $T = \SI{115+-4}{\micro\second} $ ist im Bereich der Totzeit vergleichbarer Zählrohre \cite{anleitung603}.
Die visuell bestimmte Totzeit von $T \approx 100\si{\micro\second}$ fällt dabei auch in den selben Bereich wie die Totzeit welche durch die Zwei-Quellen-Methode bestimmt wurde.


Die Zahl $Z = \SI{1.37(6)e+14}{}$ der Freigesetzte Ladung pro eingefallenem Teilchen entspricht nicht genau den Werten von $Z$ welche durch die Berechung mit der Stromstärke $I$ ermittelt wurden.
Dennoch zeigt sich, dass tatsächlich sehr viele Ladungen durch ein Teilchen freigesetzt werden.
Es entstehen also viele Townsend-Lawinen durch eine Teilchen.
Der Zusammenhang zwischen der angelegter Spannung und den freigesetzten Ladungen ist zudem linear.
