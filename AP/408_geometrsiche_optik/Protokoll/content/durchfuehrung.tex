\section{Durchführung}
\label{sec:Durchführung}
\subsection{Bestimmung der Brennweite durch Messung der Gegenstandsweite und Bildweite}
Vor der Halogenlampe, wird der Gegenstand 'Perl L' auf der optischen Bank justiert.
Das Licht soll zuerst durch das 'Perl L' strahlen und dann durch die Linse auf den Schirm fallen.
Die Höhe und Richtung der Komponenten muss genau ausgerichtet sein.
Die Linse wird hier nicht bewegt und hat einen konstanten Abstand zur Lampe.
\\
Nun wird der Gegenstand an eine bestimmte Position gebracht und der Schirm so justiert, dass ein scharfes Bild auf ihm zu sehen ist.
Die Gegenstandweite $g$ und Bildweite $b$ werden notiert.
Der Vorgang wird für $5$ weitere Gegenstandsweiten wiederholt.
\FloatBarrier

\subsection{Bestimmung der Brennweite einer Linse nach der Methode von Bessel}
Der Gegenstand und der Schirm sind fest justiert.
Der Abstand $e$ (siehe Abb. \ref{fig:bessel}) ist also konstant und sollte 4-Mal so groß wie die Brennweite $f$ der Linse sein.
\\
Nun werden die beiden Linsen verschoben bis das Bild auf dem Schirm scharf zu sehen ist.
Die Gegenstandsweite $g_1$ und Bildweite $b_1$ werden notiert und es wird eine weitere Linsenposition gesucht, bei der ein scharfes Bild auf dem Schirm zu sehen ist.
Die Gegenstandweite wird als $g_2$ und die Bildweite als $b_2$ bezeichnet und notiert.
Die Messung wird für $6$ weitere Abstände $e$ wiederholt.
\\
Als nächstes wird einmal ein blauer und einmal ein roter Filter vor dem Gegenstand befestigt.
Die oben beschriebene Messung wird für jeden Filter für jeweils drei Abstände $e$ wiederholt.
\FloatBarrier

\subsection{Bestimmung der Brennweite eines Linsensystems nach der Methode von Abbe.}
Zwischen der Halogenlampe und dem Schirm wird der Gegenstand 'Perl L', eine Zerstreuungslinse mit $f = \SI{-300}{\milli\metre}$ und eine Sammellinse mit $f = \SI{100}{\milli\metre}$ auf der optischen Bank justiert.
Die Beiden Linsen werden so nah wie möglich aneinander geschoben und der Abstand zwischen den Linsen konstant gehalten.
Die Gegenstandsgröße (Perl L) muss einmalig vermessen werden.\\
Nun wird als Punkt $A$ die Kante des ersten Linsenreiters gewählt.
Es wird ein bestimmter Abstand $g'$ zwischen Punkt $A$ und dem Gegenstand eingestellt und der Abstand $b'$ zwischen Punkt $A$ und dem Schirm gemessen.
Zudem wird die Bildgröße $B$ mit einem Lineal bestimmt.\\
Insgesamt werden für $6$ Abstände $g'$ die Größen $b'$ und $B$ vermessen.