\section{Diskussion}
\label{sec:Diskussion}
Die Geiger-Müller Charakteristik (Abb. \ref{fig:char_plot}) weist ein typisches Plateau auf.
Die Steigung des Plateau 
\begin{equation*}
    P = \SI{1.21(23)}{\frac{\percent}{100\volt}}
\end{equation*}
ist ein Maß der Qualität des Geiger-Müller-Zählrohrs.
Die Steigung ist relativ hoch und lässt auf ein nicht qualitativ hochwertiges Zählrohr schließen.
\\
Die Totzeit in Abschnitt \ref{sec:totzeit} beträgt nach der Zwei-Quellen-Methode
\begin{equation*}
    T = \SI{115(4)}{\micro\second} .
\end{equation*}
Mithilfe des Oszilloskop kann eine Totzeit von
\begin{equation*}
    T_\text{Oszi} = \SI{100(100)}{\micro\second}
\end{equation*}
ermittelt werden.
Die Abweichung beträgt $\Delta T = \SI{15}{\micro\second}$ und ist der Ablesegenauigkeit des Oszilloskop zu zuschreiben.
Insgesamt liegt die Totzeit im erwarteten Bereich.
\\
In Abschnitt \ref{sec:zaehlstrom} sind die Zahlen $Z$ ermittelt worden.
Nach Betrachtung der Abb. \ref{fig:zahl_z} kann von einem linearen Zusammenhang zwischen der Zahl $Z$ und dem Zählstrom $I$ ausgegangen werden.
Der Zusammenhang kann durch
\begin{equation}
    Z = I \cdot \SI{3.24(14)e16}{\frac{1}{Imp \cdot \ampere}} + \SI{3.0(15)e9}{\frac{1}{Imp}}
\end{equation}
angenähert werden.