\section{Diskussion}
\label{sec:Diskussion}
Das Emissionsspektrum in \autoref{sec:spektrum} entspricht der theoretischen Erwartung.
Zuerst tritt Bremsstrahlung auf (siehe Abb. \ref{fig:spektrum}, Bremsberg), dann kommt die charakteristische Strahlung (Peaks) durch Übergänge zwischen Energieniveaus der inneren Elektronenhülle hinzu.
Die theoretischen Werte stimmen mit den Messwerten überein (siehe Tab. \ref{tab:spektrum_vgl}).
\begin{table}
    \centering
    \begin{tabular}{c|ccc}
    \toprule
        & $E_\text{th} \,/\, \si{\joule}$ & $E_\text{exp} \,/\, \si{\joule}$ & Abweichung $\,/\, \%$\\
    \midrule
    $K_\alpha$-Linie & $\SI{1.2914e-15}{}$ & $\SI{1.2887e-15}{}$ & $0.21$ \\
    $K_\beta$-Linie & $\SI{1.4291e-15}{}$ & $\SI{1.4282e-15}{}$ & $0.06$ \\
    \bottomrule
    \end{tabular}
    \caption{Die experimentell gemessenen Werte und Theoriewerte \cite{klinie} im  Vergleich.}
    \label{tab:spektrum_vgl}
\end{table}
\FloatBarrier
Die Transmission als Funktion der Wellenlänge in \autoref{sec:transmission} entspricht einer linearen Funktion.
Die Messwerte weichen kaum von der linearen Funktion ab und es kann von einem linearen Zusammenhang zwischen der Trasmission $T(\lambda)$ und der Wellenlänge $\lambda$ ausgegangen werden.
\\
Die experimentell bestimmte Compton-Wellenlänge $\lambda_\text{c,exp} = \SI{3.91(7)e-12}{\metre}$ weicht um $\SI{60.91}{\percent}$ vom Literaturwert $\lambda_\text{c,th} = \SI{2.43e-12}{\metre}$ \cite{compton} ab.
Grund können Ungenauigkeiten der Messgeräte sein.
Das Geiger-Müller-Zählrohr könnte auch zusätzliche Photonen gemessen haben (z.B. Licht).
Genauere Werte können durch Abschirmung vor äußerer Strahlung erzielt werden.
\\
Bei einem sichtbaren Bereich des Spektrums ist der Versuch nicht durchführbar, da die Photonen nicht genügend Energie aufweisen um ein Elektron in ein anderes Energieniveau zu bringen.