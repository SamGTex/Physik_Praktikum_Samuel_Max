\input{header.tex}

\subject{802}
\title{Das Hooksche Gesetz}
\date{%
  Durchführung: 31.10.2019
  \hspace{3em}
  Abgabe: 05.11.2019
}

\begin{document}
\maketitle
\thispagestyle{empty}
\tableofcontents
\newpage

%\printbibliography{}
 \section{Versuchbeschreibung}
\begin{figure}
  \centering
  \includegraphics[width=\textwidth]{content/Hooksche_Gesetzt.png}
  \caption{Der Versuchsaufbau auf der \href{http://hyperion.didaktik.physik.uni-due.de/IBEs/Hooke.php}{Website}.}
  \label{fig:Versuchsaufbau}
\end{figure}
Im Versuch wird eine Feder an einem Kraftmesser angebracht. An dem losen Ende der Feder
wird dann ein Faden befestigt, der anschließend mit einer Rolle umgelenkt wird. 
Das andere Ende des Faden wird nun an einer Klammer befestigt,
die über ein Lineal beliebig verschiebar und fixierbar ist.
  \section{Versuchsdurchführung}
Zur Versuchsdurchführung wird die Klammer auf dem Lineal verschoben und anschließend fixiert.
Daraufhin wird der Wert auf dem Kraftmesser abgelesen und notiert.
Danach wird die Klammer wieder gelöst und erneut verschoben, bis die gewünschte Anzahl an Messwerten erreicht wird.
  \section{Versuchauswertung}
Durch die Durchfühnrung des Versuchs wurden folgende Werte gesammelt:
\begin{table}
  \centering
  \caption{Die Aufgenommen Daten}
  \label{tab:Messdaten}
  \begin{tabular}{c c}
  \toprule
  $ \Delta x \:/\: \si{\meter}$ & $F \:/\: \si{\newton}$ \\
  0.58 & 1.73 \\
  0.53 & 1.58 \\
  0.48 & 1.44 \\
  0.43 & 1.28 \\
  0.38 & 1.13 \\
  0.33 & 0.98 \\
  0.28 & 0.83 \\
  0.23 & 0.68 \\
  0.18 & 0.53 \\
  0.13 & 0.38 \\
  \bottomrule
  \end{tabular}
\end{table}
\FloatBarrier
Aus den aufgenommen Daten wurde nun der Mittelwert der Federkonstante $D$ wie folgt bestimmt
\begin{equation}
\sum_{i=1}^{10} \frac{1}{10} \cdot \frac{F_i}{\Delta x_i}
\end{equation}
Daraus ergibt sich
\begin{equation}
D = 2.967234488336867
\end{equation}
\end{document}
