\section{Theorie}
\label{sec:Theorie}


\subsection{Zielsetzung}

\subsection{Relaxion eines RC-Kreises}
Ein RC-Kreis wird durch aufladen oder entladen zur Relaxion gebracht.
Zum Zeitpunkt $t$ ist die Änderungsrate der betrachteten physikalischen Größe $A$ dann

\begin{equation}
    \frac{dA}{dt} = c\right [ A(t) - A(\inf) \left ] .
    \label{eqn:diffA}
\end{equation}

Aus der Gleichung folgt 

\begin{equation}
    A(t) = A(\inf) + \right[ A(0) - A(\inf) \left ] \symup{e}^{ct} .
\end{equation}

\subsubsection{Entladevorgang}

Nun wird auf den Entladevorgang eines RC-Kreises eingegangen.
Nach dem ohmschen Gesetzt gilt 
\begin{equation}
    I = \frac{U_\text{C}}{R}, 
\end{equation}
wobei $U_\text{C}$ die Spannung am Kondensator ist, welcher durch
\begin{equation}
    U_\text{C} = \frac{Q}{C}
\end{equation}