\section{Diskussion}
\label{sec:Diskussion}
\subsection{Bragg-Bedinung}
Die Überprüfung der Bragg-Bedinung ergibt, dass diese $0.2\si{\degree}$ von der theoretischen Lage abweicht.
Die experimentell bestimmte Lage befindet sich bei $\theta_\text{exp} = 28.2\si{\degree}$ wohingegen die theoretische bei $\theta_\text{theo}= 28.0\si{\degree}$ liegt.
Dies entspricht einer Abweichung von $0.7\%$.
Dies ist mit der groben Messung von $0.1\si{\degree}$-Schritten um den Peak zu brgründen.
Die Abweichung erscheint geringe, vergrößt allerdings den Fehler bei allen Rechnung die auf dieser Bedingung aufbauen.

\subsection{Emissionsspektrum der Cu-Röhre}
Die minimale Wellenlänge bzw. die maximale Energie des Bremsbergs kann nicht ermittelt werden, da die Messung in dem Bereich zu ungenau ist.
Für genauere Werte muss bis $\theta = \SI{19}{\degree}$ die Intensität in einer kleineren Schrittweite aufgenommen werden.
Für Abschnitt \ref{sec:emission} ergeben sich folgende Abweichungen zwischen Theorie- und Experimentalwert der Energien an der $K_\alpha$ und $K_\beta$-Linie: 
\begin{align*}
    \Delta E_{K,\alpha} = \SI{0.08}{\percent} \\
    \Delta E_{K,\beta} = \SI{0.21}{\percent}
\end{align*}
Die Abweichungen sind gering und können mit Messunsicherheiten der Geräte begründet werden.
\\
\subsection{Absorptionsspektren}
In der Tabelle \ref{tab:absorption} sind die berechneten Absorptionsenergien und Abschrimkonstanten sowie die recherchierten Literaturwerte aufgelistet.
Die Abweichung von experimentellem Wert zum Literaturwerten sind in Tabelle \ref{tab:abweichung} zu finden.
Die berechnete Rydbergenergie $R_{\infty,\text{exp}}= \SI{12.47(20)}{\eV}$ weicht von dem Wert welcher in der Literatur gegeben ist $R_\infty = \SI{13.6}{\eV}$ um $8.3\%$ ab.
Die Abweichung sind den experimentellen Umständen entsprechend.
Sie hätten durch kleinere Messintervalle und höhere Integrationszeiten verkleinert werden können.
Außerdem wurde die Totzeit des Geiger-Müller-Zählrohrs nicht beachtet.
\begin{table}
    \centering
    \caption{Abweichungen der berechneten Werte zum Literaturwert}
    \begin{tabular}{c c c}
        \toprule
        Element & $\Delta E\,/\,\si{\percent}$ & $ \Delta\sigma \,/\, \% $ \\
        \midrule
        Zn & 0.51 & 1.68 \\
        Ga & 0.77 & 2.77 \\
        Br & 0.06 & 0.51 \\
        Rb & 0.98 & 4.06 \\
        Sr & 0.74 & 3.00 \\
        Zr & 1.55 & 6.84 \\
        \bottomrule
    \end{tabular}
    \label{tab:abweichung}
\end{table}


\begin{table}
    \centering
    \caption{Ergebnisse und Literaturwerte der Absorptionsenergie und der Abschirmkonstante.\cite{xray}}
    \sisetup{
        table-format=2.2
    }
    \begin{tabular}{c S[table-format=1.0] S S S S}
        \toprule
        Element & $Z$ & $\frac{E}{\si{\kilo\eV}}$ & $\frac{E_\text{Lit}}{\si{\kilo\eV}}$ & $\sigma$ & $\sigma_\text{Lit}$ \\
        \midrule
        Zn & 30 & 9.60 &  9.65 & 3.63 & 3.57 \\
        Ga & 31 & 10.29 & 10.37 & 3.71 & 3.61 \\
        Br & 35 & 13.47 & 13.47 & 3.83 & 3.85 \\
        Rb & 37 & 15.05 & 15.20 & 4.10 & 3.94 \\
        Sr & 38 & 15.98 & 16.10 & 4.12 & 4.00 \\
        Zr & 40 & 17.72 & 18.00 & 4.37 & 4.09 \\
        \bottomrule
    \end{tabular}
    \label{tab:absorption}
\end{table}

