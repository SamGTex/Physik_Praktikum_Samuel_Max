\section{Durchführung}
\label{sec:Durchführung}
Zuerst wird das Beugungsbild eines einzelnen Spaltes aufgenommen.
Dazu wird der Detektor in der Mittelstellung des Messverschiebereiters fixiert.
Die Spaltblende wird so justiert, dass das Hauptmaxima im Beugungsbild auf den Detektorspalt trifft und die Nebenmaxima links und rechts vom Detektor die gleiche Intensität aufweisen.
Im Anschluss wird die Intensität $I(\xi)$ in Abhängigkeit der Detektorstellung $\xi$ aufgenommen.
Es werden $50$ Messwerte auf einem Veschiebweg von insgesamt $\SI{50}{\milli\metre}$ aufgenommen.
Das Messintervall $\Delta \xi$ wird dabei kleiner umso näher man dem Maximum kommt.
Zur Auswertung wird $I(\varphi)$ benötigt und kann nach trigonometrischen Beziehungen bestimmt werden
\begin{equation}
    \varphi \approx \tan \varphi = \frac{\xi - \xi_0}{L} \, .
\end{equation} 
Dabei ist $L$ der Abstand zwischen Beugungsobjekt und Detektorblende und $\xi_0$ die Detektorstellung für die Richtung des ungebeugten Strahls.
Zusätzlich wird noch die Breite $b$ vom Beugungsspalt abgelesen und notiert.
\\
Analog zum Einzelspalt wird die Intensitätsverteilung des Doppelspalts aufgenommen.
Zuletzt wird wieder die Spaltbreite $b$ und zusätzlich der Abstand zwischen den Spalten $g$ notiert.